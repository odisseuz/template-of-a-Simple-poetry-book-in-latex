%Preambulo
\documentclass[11pt, a5paper]{article} % {article} por escolha estilistica p livros pequenos
\usepackage[utf8]{inputenc}
\usepackage[T1]{fontenc}
\usepackage[brazil]{babel}
\usepackage{graphicx} 

%Design e Tipografia
\usepackage{mathpazo} %fonte 
\usepackage[protrusion=true,expansion=true]{microtype} 
\usepackage{xcolor} 

% Definição de Cores
\definecolor{textoSuave}{RGB}{60, 60, 60} % Cinza Chumbo
\definecolor{pretoTotal}{RGB}{0, 0, 0}      % Preto Puro

\usepackage{verse} %o pacote para poesia
\usepackage{geometry}
\usepackage{pdflscape}
\usepackage{tikz}
\usepackage[hidelinks]{hyperref}
\usetikzlibrary{decorations.text, shapes.geometric, calc}

% --- Configuração do Sumário ---
\usepackage{tocloft}
\renewcommand{\cftsecdotsep}{\cftnodot} % Remove pontilhado
\renewcommand{\cftsecfont}{\normalfont\color{textoSuave}}  % Fonte item sumario
\renewcommand{\cftsecpagefont}{\normalfont\color{textoSuave}} % Fonte numero pagina
\renewcommand{\contentsname}{\hfill sumário \hfill} % Centraliza titulo

% Funções para poemas
\newcommand{\poema}[1]{%
    \phantomsection % Cria âncora para links funcionarem
    \section*{#1}   % Título no texto (sem número)
    \addcontentsline{toc}{section}{#1} % Adiciona manualmente ao sumário
} 

%Margens
\geometry{top=1.8cm, bottom=1.8cm, left=1.8cm, right=1.8cm}

%Removendo a numeração padrão 
\pagestyle{empty}

\begin{document} %início do livro

\color{textoSuave}

% Capa
\begin{titlepage}
    \sffamily 
    \noindent
    \color{pretoTotal}

    \vspace*{2cm} 
    {\small \MakeUppercase{Nome do Autor}} \\[-0.2cm] 
    \rule{3cm}{4pt} 
    \vspace*{\fill} 

    {\fontsize{50}{55}\selectfont \bfseries 
    TÍTULO \\ 
    DA \\ 
    OBRA}

    \vspace{1.5cm}

    \begin{tikzpicture}
        \fill[black] (0,0) circle (1.2cm);
    \end{tikzpicture}

    \vspace*{\fill}

    {\footnotesize Fortaleza, 2026} 
    
    \rmfamily 
\end{titlepage}

% PÁGINA 2: FICHA TÉCNICA
\newpage
\thispagestyle{empty}

% Espaço reservado para o texto técnico (ISBN, etc)
\vspace*{\fill}
\begin{flushleft}
    \footnotesize
    \textbf{Ficha Técnica} \\
    Autora: Taz\\
    Ano: 2026 \\
    ISBN: [Em breve] \\
    \textit{Todos os direitos reservados.}
\end{flushleft}

% --- SUMÁRIO ---
\newpage
\tableofcontents 
\thispagestyle{empty} 
\newpage

% --- INÍCIO DOS POEMAS ------

% exp com 2 poesias na msm pagina

\pagestyle{plain} 
\setcounter{page}{1}

\poema{abbububublé}

\begin{verse}
    abububule, abubububle \\
    auaubububu, bububuele \\
\end{verse}

\vspace{2cm}

\poema{bubublé}

\begin{verse}
    abu\\
    bu\\
    blé\\
    \vspace{0.5cm} %espaço entre poesias
    o bububule\\
    abububbule
\end{verse}

% --- POEMA 3 ---
\newpage
\poema{ipsum ababububle} 

\vspace*{\fill} 
\begin{verse}
    laralralrl\\
    asdkadasdlk\\
    aaaaaaaaaaaaaaaaaaaaaaaaaaaaaaaaa\\

    \vspace{0.5cm}
    
    adsasdçkãsdçkasdçlk\\
    
    \vspace{0.5cm}

    adçlasdçladsk\\
\end{verse}
\vspace*{\fill}

%contra-capa
\newpage
\vspace*{\fill} 

\begin{center}
    % Dictionary style
    \begin{minipage}{10cm}
        \centering
        \textbf{\large afeto} \\
        \textit{\small (substantivo)} \\
        \vspace{0.2cm}
        \footnotesize
        1. Disposição da alma em ser tocada pelo mundo, \textit{vir-a-ser}. \\
        2. Alimento invisível tão vital quanto o pão; manifesta-se no beijo, no abraço e no silêncio entre o eu e o Outro. \\
        3. A arte de transformar a presença em permanência na vida interna.
    \end{minipage}

    \vspace{1.5cm}
    
    $\diamond$ 
    
    \vspace{1.5cm}
    
    % aaaaaaaaaaaaaaaaa
    \vspace{0.3cm}
    \begin{minipage}{8cm}
        \centering \footnotesize \itshape
        \color{textoSuave}
        Por toda vida e coragem derramadas nestas folhas de papel.
    \end{minipage}
    
    \vspace{3cm} 
    
    % O RODAPÉ TÉCNICO
    {\tiny \color{textoSuave} 
    Edição 001 $\cdot$ Fortaleza/CE \\
    Design e Tipografia por @desenhosesquisitos123 \\
    Fevereiro de 2026 \\
    Composto em \LaTeX
    }
\end{center}


\end{document}
